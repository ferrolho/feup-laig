\documentclass[a4paper]{article}

\usepackage[portuguese]{babel}
\usepackage[utf8]{inputenc}
\usepackage{indentfirst}
\usepackage{graphicx}
\usepackage{verbatim}
\usepackage{caption}
\usepackage{subcaption}
\usepackage{wrapfig}
\usepackage{ragged2e}
\usepackage{listings}
\usepackage{float}

\begin{document}

\setlength{\textwidth}{16cm}
\setlength{\textheight}{22cm}

\title{\Huge\textbf{Eximo - LAIG3}\linebreak\linebreak\linebreak
\Large\textbf{Manual de Utilizador}\linebreak\linebreak
\linebreak\linebreak
\includegraphics[scale=0.1]{res/feup-logo.png}\linebreak\linebreak
\linebreak\linebreak
\Large{Mestrado Integrado em Engenharia Informática e Computação} \linebreak\linebreak
\Large{Laboratório de Aplicações com Interface Gráfica}\linebreak
}

\author{\textbf{T6G10: Eximo}\\
Henrique Manuel Martins Ferrolho -  201202772\\
João Filipe Figueiredo Pereira - 201104203 \\
Maria João Pombinho Miranda - 201204026 \\
\linebreak\linebreak \\
 \\ Faculdade de Engenharia da Universidade do Porto \\ Rua Roberto Frias, s\/n, 4200-465 Porto, Portugal \linebreak\linebreak\linebreak
\linebreak\linebreak\vspace{1cm}}

\maketitle
\thispagestyle{empty}

%************************************************************************************************
%************************************************************************************************

\newpage

%Todas as figuras devem ser referidas no texto. %\ref{fig:codigoFigura}
%
%%Exemplo de código para inserção de figuras
%%\begin{figure}[h!]
%%\begin{center}
%%\includegraphics[height=1.5cm,width=1.5cm]{white_piece.png}
%%\caption{Peça Branca - Jogador 1}
%%\label{fig:1}
%%\end{center}
%%\end{figure}
%%\includegraphics[scale=0.5]{path relativo da imagem}
%%\includegraphics[scale=•]{•} relativo da imagem}

%%\begin{figure}[h!]
%%\centering
%%\includegraphics[height=1.5cm,width=1.5cm]{white_piece.png}
%%\caption{Peça Branca}
%%\label{fig:whitePiece}
%%\end{figure}

%%\hfill

%%\begin{figure}[h!]
%%\centering
%%\includegraphics[height=1.5cm,width=1.5cm]{black_piece.png}
%%\caption{Peça Preta}
%%\label{fig:blackPiece}
%%\end{figure}

%%\hfill

%%\begin{figure}[h!]
%%\centering
%%\includegraphics[height=3cm,width=3cm]{board.png}
%%\caption{Tabuleiro}
%%\label{fig:board}
%%\end{figure}




%
%
%\textit{Para escrever em itálico}
%\textbf{Para escrever em negrito}
%Para escrever em letra normal
%``Para escrever texto entre aspas''
%
%Para fazer parágrafo, deixar uma linha em branco.
%
%Como fazer bullet points:
%\begin{itemize}
	%\item Item1
	%\item Item2
%\end{itemize}
%
%Como enumerar itens:
%\begin{enumerate}
	%\item Item 1
	%\item Item 2
%\end{enumerate}
%
%\begin{quote}``Isto é uma citação''\end{quote}

\newpage

%%%%%%%%%%%%%%%%%%%%%%%%%%
\section{O Jogo Eximo}
%%Descrever detalhadamente o jogo, a sua história e, principalmente, as suas regras.
%%Devem ser incluidas imagens apropriadas para explicar o funcionamento do jogo.
%%Devem ser incluidas as fontes de informação (e.g. URLs em rodapé).

\large{\textbf{História}}
\begin{small}

Eximo é um jogo de tabuleiro da família das Damas, concebido em 1 de Fevereiro de 2013.\newline
\end{small}

\large{\textbf{Detalhes do Jogo}}
\begin{small}

O jogo realiza-se num tabuleiro de dimensões 8x8, em que as casas têm todas cores semelhantes. Cada jogador começa com 16 peças colocadas em locais pré-definidos no respectivo lado do tabuleiro, como mostra a imagem abaixo.\newline

\begin{figure}[h!]
  \begin{minipage}[h!]{0.2\textwidth}
    \includegraphics[width=0.4\textwidth]{res/white_piece.png}
    \centering
    \caption{Peça branca}
    \label{fig:2}
  \end{minipage}
	\quad\quad\quad
  \begin{minipage}[h!]{0.2\textwidth}
    \includegraphics[width=\textwidth]{res/board.png}
    \caption{Tabuleiro}
    \label{fig:3}
  \end{minipage}
	\quad\quad\quad
  \begin{minipage}[h!]{0.2\textwidth}
    \includegraphics[width=0.4\textwidth]{res/black_piece.png}
	\centering
    \caption{Peça preta}
    \label{fig:4}
  \end{minipage}
\end{figure}

No jogo, as movimentações e as capturas podem ser ortogonais ou diagonais. Há apenas um tipo de peça: os homens. Os homens podem saltar sem efectuar captura. Quando um homem atinge a última linha, ocorre a libertação de outro homem.
\end{small}\newline

\large{\textbf{Objectivo}}
\begin{small}

O objectivo do jogo é, tal como nas Damas, \textbf{capturar todas as peças} do oponente, saltando sobre elas, ou \textbf{incapacitar o adversário} de realizar qualquer movimento.
\end{small}\newline

\large{\textbf{Jogada}}
\begin{small}

Em cada jogada, um jogador pode fazer uma de duas acções: \textbf{mover ou capturar}.
\end{small}\newline

\large{\textbf{Movimento}}
\begin{small}

Uma peça pode mover-se em três direcções: para a frente ou na diagonal (\textbf{norte, nordeste ou noroeste}). Numa jogada, o movimento nunca pode ser efectuado para a retaguarda. 

Existem dois tipos de movimentos: \textbf{Normal e Salto}. 
\begin{itemize}
\item \textbf{Movimento Normal}:
uma peça move-se para uma \textbf{casa adjacente e vazia}. 
\item \textbf{Movimento em Salto}:
\textbf{uma peça salta sobre uma peça aliada adjacente, se e só se a casa correspondente} (ao lado da peça aliada) \textbf{estiver vazia}, colocando assim a peça nessa casa. Se a mesma peça do jogador puder continuar a realizar o mesmo movimento de salto sobre outra peça amigável então terá de o fazer. \textbf{Durante um movimento de salto a peça não pode capturar peças inimigas}.

Quando existe \textbf{mais do que uma forma de saltar}, o jogador \textbf{pode escolher a peça que irá usar para executar o salto}, bem como o tipo de salto ou sequência de saltos a fazer. Não é obrigatório que a sequência de saltos escolhida pelo jogador seja aquela que possui o maior número de saltos; porém, \textbf{após escolher uma sequência, o jogador deve executar todos os saltos possíveis}.
\end{itemize}
\end{small}

\begin{figure}[h!]
  \begin{minipage}[h!]{0.3\textwidth}
    \includegraphics[width=\textwidth]{res/normalMove.png}
    \caption{Movimento Normal}
    \label{fig:5}
  \end{minipage}
\quad\quad\quad\quad\quad\quad\quad\quad
  \begin{minipage}[h!]{0.3\textwidth}
    \includegraphics[width=\textwidth]{res/jumpMove.png}
    \caption{Movimento em Salto}
    \label{fig:6}
  \end{minipage}
  \caption{Movimentos}
\end{figure}

\large{\textbf{Captura}}
\begin{small}

Um \textbf{jogador pode capturar} em cinco direcções: \textbf{frente}, \textbf{diagonal para a frente}, \textbf{direita} ou \textbf{esquerda} (norte, nordeste, noroeste, este ou oeste). 

\begin{itemize}
\item \textbf{Captura}:
um jogador \textbf{salta sobre uma peça adjacente do adversário}, se a \textbf{próxima casa}, na mesma direcção, \textbf{estiver vazia}, colocando, assim, a peça sobre essa casa. A \textbf{peça do oponente é removida do tabuleiro}. Se a \textbf{peça} do mesmo jogador \textbf{puder continuar a capturar outras peças do adversário}, então \textbf{deve fazê-lo}. A \textbf{captura é obrigatória} e deve continuar enquanto for possível.
\end{itemize}

\begin{figure}[h!]
  \begin{minipage}[h!]{0.3\textwidth}
    \includegraphics[height=5cm,width=5cm]{res/captureMove.png}
    \caption{Estado anterior à captura}
    \label{fig:7}
  \end{minipage}
	\quad\quad\quad\quad\quad\quad\quad
  \begin{minipage}[h!]{0.3\textwidth}
    \includegraphics[height=5cm,width=5cm]{res/captureMoveResult.png}
    \caption{Estado posterior à captura}
    \label{fig:8}
  \end{minipage}
\end{figure}

Tal como no \textbf{Movimento de Salto}, \textbf{o jogador escolhe} livremente \textbf{qual a sequência de saltos a efectuar}.
\end{small}\newline

\large{\textbf{Última Linha}}
\begin{small}

Quando uma peça atinge a extremidade do tabuleiro, essa peça é removida de imediato e o jogador recebe dois movimentos extra para efectuar nesse mesmo momento: colocar duas peças novas numa casa vazia localizada nas duas primeiras linhas, à excepção das quatro casas laterais (duas do lado esquerdo, e duas do lado direito).
\end{small}\newline


\section{Compilação do Programa}
\large{\textbf{Eximo - PROLOG}}

Para compilação do programa é necessário, antes de tudo, a instalação do Sicstus para executar o programa desenvolvido em PROLOG. Tendo este passo realizado, abrimos um terminal onde compilamos o ficheiro \textbf{server.pl} da seguinte forma:

\textbf{\textit{consult('caminho-para-o-ficheiro').}}

seguido da sua execução:

\textbf{\textit{server.}}

Neste ponto o programa já se encontra à espera de uma resposta para ligação com outro.

\hfill

\large{\textbf{Eximo - C++/OpenGL}}

Para executar a aplicação desenvolvida em C++ basta correr o executável com o ficheiro .ANF já presente.

\textbf{\textit{./LAIG3 "caminho-para-o-ficheiro-eximo.xml"}}

A conexão é deste modo feita, estando o programa a correr normalmente. Para a aplicação das texturas é necessário a pasta das mesmas estar junto ao executável.

\section{Interação do Utilizador}
O jogo de momento tem poucas funcionalidades. Para mover uma peça basta fazer um clique em cima da mesma e, seguindo as regras, clicar numa casa de destino. Os movimentos de captura são idênticos aos de movimento.

\pagebreak

\end{document}
